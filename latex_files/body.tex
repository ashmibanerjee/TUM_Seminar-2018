\section{Introduction}
		
Tourism industry has always attracted a lot of research owing to its diversity. A traveller\textquotesingle s choice of destinations, search for information before, during and after his trip, followed by his social media usage to talk about his travels have been found to be highly correlated with his personality. Hence, in other words, it can also be said that tourism destinations can also be mapped to his personality, thus providing him tourism recommendations of his choice. 		

This paper can be divided into the following sections. The first section, \ref{1} analyses the traveller\textquotesingle s use of social media to talk about his trip. Owing to the Internet\textquotesingle s virtual capabilities, it can include a plethora of physical sources of information and it can provide timely and accurate information relevant to travellers not only by pictures but also by videos and sounds, making it more inexpensive and convenient than traditional means \cite{amaro2016travelers}. This existing information can be utilized by the traveller in order to research about his trip and every individual\textquotesingle s use of social media varies according to his personality. Clustering of the personalities can be performed with respect to his social media usage to come up with different categorizations for traveller types.
				
The second part, section \ref{2} tries to establish the factors that governs a traveller\textquotesingle s engagement in travel related social media. It is found that personality of the traveller plays a massive role in determining how likely he is to indulge in social media usage and generate content for other travellers on social media. Besides, a multitude of other factors like the age, sex and demographics are also involved in this process.


It is essential to have a good understanding of a traveller\textquotesingle s personality in order to elicit his preferences for tourist destinations and other travel recommendations. Section \ref{3} illustrates the relation of personality to the choice of vacation destinations, leisure activities and also other travel-related decisions. Various tourist roles can be categorized to describe the relationship between an individual\textquotesingle s travel behaviour, their preferences, interests and requirements and in turn this information can be used to design travel recommender systems. While Neidhardt et.al \cite{neidhardt2015picture}\cite{neidhardt2014eliciting} combined the personality traits (long-term behaviour) and 17
tourist roles (short-term behaviour), to introduce a seven-factor
model by conducting factor analysis, this seven factor model was later used by Sertkeran et. al to map several tourist destination features in section \ref{4} \cite{sertkan2018mapping}.





% \subsection{How is the traveller using social media to talk about his trip?}
% \section{Traveller\textquotesingle s use of social media to talk about his trip}\label{1}

\section{Social Media and Travel}

The evolution of internet led to an increase in travellers using the social media for their travel. Travellers now can be provided with timely, accurate information not only in the form of pictures but also by videos and sounds- all relevant to their travel. Hence, the social media boom was accorded as the most influential and powerful factor while planning of trips and decision making during travels, eventually playing a major role in the traveller\textquotesingle s overall travel experience.

The social media and its relationship with travellers and the travel community can be divided into the following sub-sections. While subsection \ref{2.1} elaborates the importance of social media in the travel context, subsection \ref{2.2} brings out the various phases of use of social media in a traveller's life. The last subsection \ref{2.3} looks into the demographic and psychological aspects governing a traveller and his use of social media.

\subsection{Importance of social media in the travel context}\label{2.1}

The evidence of social media\textquotesingle s importance in the travel context from different sources can be summarised as the following. While social media websites in the United Kingdom are the main resource for planning a holiday, 44\% of leisure travellers in Asia Pacific region use social media platforms to seek advice and as inspiration regarding their choice of travel destinations \cite{eMarketer, 2013}\cite{amaro2016travelers}. According to a study performed by TripAdvisor\textquotesingle s TripBarometer, online travel reviews influence 89\% of the travellers around the globe while they are choosing their accommodation. Furthermore it also shows that more than 50\% of the travellers actually change their original travel plans after browsing through social media websites\cite{amaro2016travelers}.The study by TripAdvisor cited in the paper by Amaro et. al also showed that 96\% of hoteliers claim that the reviews are influential while generating bookings. 50\% of travel companies have asserted that direct bookings have been generated from social media while consumer generated content influences 10 billion dollars in online travel bookings \cite{amaro2016travelers}.

A significant number of studies have shown that in addition to the informational benefits of social media, reading travel reviews made the trip planning process fun and enjoyable and got the travellers more excited about travelling \cite{10.1007/978-3-211-77280-5_4}\cite{gretzel2007online} \cite{parra2012travellers}. Users from online travel communities such as TripAdvisor.com or VirtualTourist.com indulged in online community activities for informational as well as hedonic benefits \cite{chung2008web}. The study by Cusick in 2014 pointed out that the hedonic needs played an important role in predicting the level of participation of a traveller in an online travel community \cite{doi:10.1177/0047287503258824}. Positive relationships have been established between the perceived hedonic benefits and the intention to use social media for travel planning by Ayeh et. al \cite{ayeh2013predicting}.  Perceived enjoyment also can be used as a determinant of creating travel content online. It was found by Yoo and Gretzel in 2011 that enjoyment is capable of driving the creation of travel content generated media\cite{YOO2011609}. A more recent study by Kang and Schuett in 2013 affirmed the above conclusion by stating that perceived enjoyment not only had a positive relationship with the use of social media for travel planning but also affected the actual creation of travel content online \cite{doi:10.1080/10548408.2013.751237}.
All these studies show empirical evidence of how individuals use travel-related social media for both information and hedonic purposes. The personalization of information search, with the advent of Web 2.0 also contributes to its hedonic value \cite{doi:10.1080/14616688.2012.762542}.

\subsection{Traveller\textquotesingle s use of social media to talk about his trip}\label{2.2}

A traveller\textquotesingle s use of social media can be divided into three phases: before, during and after the trip. It was found that travellers use social media extensively to research about his trip, read more about the destinations and viewing the user generated content (UGC). However, they are not found to be actively participating by creating content. This phase hence mostly involves searching for ideas for probable destinations, planning the trip itinerary, looking up options for accommodation, and opportunities for excursions and other leisure activities\cite{cox2009role}; \cite{fotis2012social}. The use of social media before travelling helps to generate more ideas, and helps to imagine how places will be like, reducing the possibilities of risks involved in the process.

Studies showed that a traveller\textquotesingle s use of social media for travel purposes, during the trip is much lower than its use before the trip \cite{cox2009role}; \cite{fotis2012social}. While Fotis et.al shows that 30\% of the respondents searched for travel-related information during their vacations, in Cox et al. the percentage was seen to be reduced to 6\%\cite{fotis2012social}\cite{cox2009role}. During this period, travellers not only search for travel-related information but also share information regarding their travel with the help of photos, videos, reviews etc \cite{amaro2016travelers}. However, the amount of social media content produced during this phase has been found to be much lower than the consumption of social media resources \cite{fotis2012social}. It can be found that social media geo-location sites such as Foursquare provides discounts and coupons to travellers by encouraging them to use social media during this phase, thus fostering tourism and hospitality businesses \cite{hudson2013impact}.

Travellers use social media after the trip to post information regarding their trip. This can be in the form of comments, reviews, photos or pictures \cite{fotis2012social}\cite{parra2012travellers}. This content is known as the user generated content(UGC) \cite{simms2012online}. Social media producing takes place during this phase. This production encompasses the creation and publication of an individual\textquotesingle s personal contents like texts, images, audio and video \cite{shao2009understanding}. 


\subsection{Categorization of travellers based on their social media usage}\label{2.3}

Several studies performed a cluster analysis to identify the different segments of travellers\cite{chiu2012china}\cite{foster2011exploring}\cite{kurtulucs2015social} . It identified three broad groups - the social enthusiasts, the participators and the inactives. Specifically, the study by Amaro et. al found five segments that are capable of distinguishing social media users according to their degree of involvement in consumption and generation of travel contents \cite{amaro2016travelers}. Even though no particular difference has been found among these segments based on gender or national travel experience, however, the number of international trips has been found to be significantly lower in the inactive cluster. This study also reports a higher percentage of travellers who purchase travel online in the segments characterized by more involvement with travel content on social media. Significant differences were also spotted with respect to other aspects, namely, age, perceived enjoyment and social media involvement. 	The percentage of travellers using online websites to purchase travel services is smaller in the inactive segment (51\%), followed by segment 2 and 3 (60\% and 62\%, respectively), and finally segments 4 and 5 with 70\% and 69\%, respectively.

These results provide useful insights to travel online marketers and social media websites providers who need to be aware of the different segments in order to customize their websites accordingly \cite{amaro2016travelers}.

It can be concluded from the study by Amaro et.al in 2016 that fully engaged social media users, as well as occasional consumer and creators, the two segments with a higher level of social media creation, are comprised of younger travellers who perceive higher levels of enjoyment with the use of social media for travel purposes and are also more involved with social media websites. The importance of these segments to online travel providers can be accorded to their participation in creating travel related content, to influence others and, in turn, affect travel decisions \cite{amaro2016travelers}.

Numerous studies have shown that the people consuming travel information are significantly higher than people generating it \cite{pan2012theoretical} \cite{YOO2011609}.  

While it was found that creation of online travel content increased with education level until university level and then decreased  \cite{ip2012profiling}, Yoo in 2016 found no significant differences, between travel social media creators and non- creators, in terms of education \cite{yoo2016use}. The findings of Amaro et.al in 2016 however clarify this issue by attributing the correlation between the higher education level, doctoral degree, and lesser creation of travel content to ageing and time restrictions, (as higher educational levels are associated with older age classes and higher level jobs which in turn is associated with lower levels of social media creation)\cite{amaro2016travelers}. 

		


% \subsection{What are the factors behind a travellers’ engagement in travel-related social media?}
\section{Factors governing a traveller\textquotesingle s engagement in travel-related social media}\label{2}
The bridge between the number of users and the number of actual content creators remains large even though there has been an increase in the number of travellers engaging themselves in consumer-generated media (CGM) use and creation, making it important to find out the driving force behind this minority of creators and what makes them different from those who only use CGM. \cite{yoo2011influence} 

\subsection{Socio-demographic factors and cultural influence}

 

The study by Youcheng Wang and Daniel R. Fesenmaier in 2004 demonstrated that levels of participation in online tourist communities can be explained in terms of the fulfillment of needs. They proposed that tourists who participate in online tourist communities are motivated by functional, social and psychological needs. While travellers collect and consume information to satiate their functional needs, they interact with the other members of the community and build relationships in order to satisfy their social needs. They meet their psychological needs by making the community a part of their life and engaging actively in various activities pertaining to building relationships and other creative forms of communication \cite{wang2002defining}.

Some studies found that the people\textquotesingle s age, gender, income, educational levels and race have a profound impact on their CGM use and creation ((e.g.\cite{lenhart2008teens}\cite{verna2009user}. While Verna's work in 2009 found a close link between CGM creation and consumption and suggesting that CGM is more actively used and created by younger users \cite{verna2009user}, the study by Lenhart et.al in 2008 stated that, younger people are generally more active users for most types of CGM \cite{lenhart2008teens}.  Technorati's research in 2009 focused on the fact that US bloggers were mostly male and predominantly 35 or older in age \cite{Technorati, 2008}. Remarkable gender differences were noticed in other studies too. For example, US males tend to dominate the females in CGM content activity among adult demographics, while females tend to outnumber when samples are limited to preteens, teens and college students \cite{verna2009user}. A recent demographic profile report by \cite{eMarketer, 2009} says that US CGM users are more likely college educated, full-time employed and dominantly white. In addition to this, a variety of people were found to use different social networking sites. Professional networking sites like LinkedIn users tend to be more educated, having a higher income and are more likely employed full-time while social networking sites like Facebook and MySpace users were found to have lower incomes and are more likely students \cite{eMarketer, 2009}. In the travel CGM context, usage and creation of travel reviews are one of the most popular CGM activities \cite{gretzel2007online}.
 

Considering travel-related CGM creation, Yoo and Gretzel (2008) demonstrated that travellers\textquotesingle gender and income level also influences his motivations to write travel review online \cite{yoo2008motivates}. Other variables influencing travellers\textquotesingle CGM use and creation can be attributed to travelers\textquotesingle nationality, culture, membership in a generational cohort, suggesting that travelers’ personal characteristics are important factors to be considered when trying to understand engagement with CGM \cite{yoo2011influence}.



If we look specifically at the creation of CGM for adult demographics, a number of studies have proposed that CGM creators are more likely to be young, college students and male \cite{yoo2011influence}.
Considering travel CGM, similar trends were also spotted. For instance, travel CGM creators tend to be young males,with higher levels of incomes and greater internet skills \cite{yoo2008understanding}. They are also more likely to be frequent travellers as well as highly involved in trip planning \cite{gretzel2008use}. Moreover, being part of a collectivist culture was reported to increase the probability to produce contents targeting a general audience, while individualistic values lead to the creation of contents that reflect personal experiences and thus serving the purpose of documentation and ego-enhancement \cite{lee2009social}. In addition to this, posting travel photos was found to be more prominent in younger generations while boomers and seniors are less likely to post photos online \cite{yoo2011influence}.

% \subsection{What role does personality of the traveller play in elicitating his preferences on social media?}
    
\subsection{Influence of travellers\textquotesingle  personality in elicitating his preferences on social media}\label{3}



Even though numerous personal characteristics have been examined in consumer behaviour research, a traveller\textquotesingle s personality has been found to be a particularly influential trait that is capable of predicting behaviour over time and across various situations \cite{woszczynski2002exploring}. Several studies demonstrate that personality of the traveller plays a significant role in predicting different online behaviors\cite{yoo2011influence} \cite{acar2007online}; \cite{tuten2001understanding}. Hence, considering that personality has been found to be an important factor influencing a wide variety of human behaviors and choices, it is necessary to examine its impacts in the context of CGM \cite{yoo2011influence}. In tourism research, personality has often been used as a platform for market segmentation purposes, with Plog’s travel personality types along an allocentrism\textemdash psychocentrism continuum being a very important research\cite{plog1973destination}\cite{yoo2011influence}.

The studies by Madrigal, 1995; Nickerson et.al, 1991; Roehl et.al, 1992 suggest that personality is related to the choice of vacation destinations, leisure activities and also other travel-related decisions \cite{madrigal1995personal}\cite{nickerson1991traveler}\cite{roehl1992risk}. Madrigal, 1995, one of the most significant ones from the aforementioned list, examined the relationship between the List of Values (LOV) and Plog\textquotesingle s traveler personality type scale and each of their ability to predict travel style. They collected survey data from a convenience sample of 514 visitors to a tourist destination in Arizona. Results demonstrated that personal values were significantly related to traveler personality type (p < .001). Furthermore, it was seen that personal values significantly differentiated group travelers from independent travelers (p < .001), while on the other hand, Plog\textquotesingle s scale was unable to do so (p > .25). It was hence concluded that Plog\textquotesingle s measure of traveler personality type may more accurately be conceptualized in terms of locus of control \cite{madrigal1995personal}.

The influence of a traveller's personality can be analyzed on the 
\subsubsection{On CGM creation behaviours: }

A number of other studies also provide significant correlation between personality and CGM creation behaviors. Tuten and Bosnjak (2001) examined the influence of personality, specifically the Five Factor Model of Personality \cite{goldberg1990alternative}, on the use of web and found that while openness is positively related to using the Internet for entertainment and searching for information about products, neuroticism is negatively related to Web usage. Several research papers support a significant association between personality and knowledge sharing intentions\cite{cabrera2006determinants}\cite{wang2007personality},\cite{matzler2011personality}\cite{yoo2011influence}. Even though Cabrera et.al found a positive correlation between the three dimensions of the Big Five personality model \cite{goldberg1990alternative} and knowledge sharing intentions, other papers argued that agreeableness can be accorded for more sharing knowledge with others as agreeable people tend to be more collaborative than competitive. They also described that conscientiousness can be correlated to knowledge sharing intentions as conscientious people tend to show a non-reluctance to contribute to community success by knowledge sharing (\cite{matzler2011personality},  \cite{barrick1991big}, \cite{liao2004multilevel}).They also described that conscientiousness can be correlated to knowledge sharing intentions as conscientious people tend to show a non-reluctance to contribute to community success by knowledge sharing. Similarly, Wang and Yang (2007) affirms that extraversion, agreeableness and conscientiousness are positively related to individuals’ intentions to share knowledge\cite{wang2007personality}.


\subsubsection{On Social Media context: }

A substantial amount of research has been conducted to study the influences of personality in social media context. Acar and Polonsky (2007) examined the impact of extroversion in terms of online social network use and found that extraverts nurture larger social networks\cite{acar2007online}. Relationships between personality and motivations for playing online games were explored by Jeng and Teng (2008)\cite{jeng2008personality}. Their results demonstrated that while openness was positively correlated to discovery and role-playing motivations, conscientiousness was positively associated with escapism motivations. Furthermore, while extraversion was found to increase the teamwork motivations agreeableness provided advancement motivations in contrary to neuroticism, which was found to be negatively related to teamwork motivations. Ross et. al (2009) specifically investigated how the Five-Factor Model of personality can be related to the use of Social media like Facebook and CGM creation. Their findings indicate that certain aspects of Facebook usages and personality variables share a high correlation. Extraverted individuals were found to be belonging to significantly more Facebook groups even though their difference in number of Facebook friends with their introverted counterparts were not significantly high. Neuroticism played a role in information control with more neurotic users being more likely to prefer communication through the Facebook Wall, on which the information can be more easily controlled. People with high traits of Openness showed a higher likeliness to be sociable through Facebook\cite{ross2009personality}. 


\subsubsection{On Travel context: }

The study by Wang and Fesenmaier (2003) investigated the influence of individuals\textquotesingle personality in the context of an online travel community. They influence of members\textquotesingle active personality on their level of actual contributions to the online travel community. The results indicated that people with active personalities were more probable to have higher levels of online contributions in contrary to the people who have less active personalities.


A lot of research has already been conducted to identify and categorize various tourist roles and describing the relationship between an individual\textquotesingle s travel behaviour, their preferences, interests and requirements. Gibson and Yiannakis (2002) found out 17 different tourist roles, capturing short-term behaviour\cite{gibson2002tourist}). Gretzel et. al (2006) elucidated that tourist roles can be used to recommend touristic activities and in turn destinations\cite{gretzel2006travel}. Delic, Neidhardt and Werthner (2016) provided significant evidence to the relation between the Big-Five personality traits \cite{goldberg1990alternative} and the 17 tourist roles proposed by Gibson \& Yiannakis, (2002) \cite{delic2016sun}. According to Woszczynski, Roth, \& Segars, 2002, personality traits tend to stabilize over time and can thus be considered as long-term preferences of a person \cite{woszczynski2002exploring}.

Combining the personality traits (long-term behaviour) and 17 tourist roles (short-term behaviour), Neidhardt et al. (2014, 2015) introduced a seven-factor model (\ref{table: seven factor model}) by conducting factor analysis\cite{neidhardt2015picture} \cite{neidhardt2014eliciting}. These factors namely- Sun \& Chill-Out, Knowledge \& Travel, Independence \& History, Culture \& Indulgence, Social \& Sport, Action \& Fun, and Nature \& Recreation form a seven-dimensional vector space and refer to the travel behavioural pattern of a traveller. They are also easier to process computationally as well as cognitively in contrary to the original 22 dimensions. Depending on a picture-selection process, a user\textquotesingle s profile is determined. This user profile is composed of a score for each of the seven factors and thus can be thought as a point in the seven-dimensional vector space. To provide recommendations to a user, the items that are closest or most similar to him/her are determined . Thus, also the items have to be mapped into the vector space, i.e., represented with respect to the travel behavioural patterns. However, more than 10,000 tourism products had to be initially mapped manually by experts in order to build up a reasonable recommendation base making this approach not scalable.


Sertkan et. al, (2018) aims to find an automated way of relating tourism products and travel behavioural patterns \cite{sertkan2018mapping}. They examined the relationship between the seven-factor model \cite{neidhardt2015picture}\cite{neidhardt2014eliciting}  and the attributes of destinations in order to map them correctly and cluster similar destinations together for the purpose of a better understanding of the scenario and generalization.


\begin{table*}[t]
    \centering
    \resizebox{\textwidth}{!}{%
    \begin{tabular}{p{8cm}p{8cm}}
         \hline
         \textbf{Factor} & \textbf{Description}  \\ [0.5ex] 
         \hline\hline
         Sun \& Chill-Out & A neurotic sun lover, who likes warm weather and sun bathing and does not like cold, rainy or crowded places  \\ [1ex]
         \hline
         Knowledge \& Travel & An open minded, educational and well-organized mass tourist, who likes traveling in groups and gaining knowledge, rather than being lazy  \\
         \hline
         Independence \& History & An independent mass tourist, who is searching for the meaning of life, is interested in history and tradition, and likes to travel independently, rather than organized tours and travels \\
         \hline
         Culture \& Indulgence & An extroverted, culture and history loving high-class tourist, who is also a connoisseur of good food and wine \\
         \hline
         Social \& Sport & An open minded sportive traveller, who loves to socialize with locals and does not likes areas of intense tourism  \\ [1ex] 
         \hline
         Action \& Fun & A jet setting thrill seeker, who loves action, party, and exclusiveness and avoids quiet and peaceful places \\ [1ex] 
         \hline
         Nature \& Recreation & A nature and silence lover, who wants to escape from everyday life and avoids crowded places and large cities  \\ [1ex] 
         \hline
        
        \end{tabular}}
    \caption{Seven Factor Model \cite{neidhardt2014eliciting}\cite{neidhardt2015picture}}
    \label{table: seven factor model}

\end{table*}

% \section{Mapping of Destination Features to Seven Factors}\label{4}

% The study by Sertkan et.al did not only to project destinations into the seven-dimensional vector space of travel behavioural patterns using their features, but more importantly tried to establish a relationship between the seven-factors and destination features \cite{sertkan2018mapping}.
% Gareth et.al proposed the choice of linear models over more complex ones when targeting inference and interpretability \cite{james2013introduction}. Considering this fact, a multitude of linear regression models with step-wise variable selection was applied \cite{james2013introduction} \cite{sertkan2018mapping}. 

% As the seven factors are obtained from factor analysis, they are considered independent from each other. Hence, for every travel behavioural pattern, a separate model can be fitted. This model takes the features of a destination as input and returns the factor score (0–1) as output.
% The train-test split of the data-sample is done at 80/20 and model performance is assessed by R2 (proportion of variance explained) and root mean square error (standard deviation of the residuals/prediction errors) of training (RMSE\textsubscript{train}) and test set (RMSE\textsubscript{test})\cite{sertkan2018mapping}.

% The performance of each model has been shown in table \ref{fig:table3}.

% \begin{figure}
%   \includegraphics[width=\linewidth]{latex_files/figures/table3.png}
%   \caption{Performance of the resulting multiple linear regression models \cite{sertkan2018mapping}}
%   \label{fig:table3}
% \end{figure}

					
% The resulting models provide strong evidence that there lies a statistically significant relationship between selected destination features and the seven factors,(used in the corresponding models), with p < 0.001. The values of  RMSE\textsubscript{train} and RMSE\textsubscript{test} are close, indicating that the performance of the resulting models will be similar out of sample\cite{sertkan2018mapping}.  
					
% Overall, all travel behavioural patterns are well described by the resulting models(52\textemdash76\% of the variance), except Social \& Sports, where only 19\% of the variance is explained. This can be accorded to the uneven distribution of the expert mapping of Social \& Sports, where 53.83\% of the destination scored with 0.5 and only 1.78\% scored with 0 and 4.10\% with 1 respectively.

% However, there is significant evidence of a relationship between the destination features and the factor Social \& Sports. 70 and 76\% of the variance can be explained respectively for Sun \& Chill-Out and Action-Fun, making them the best fitted models\cite{sertkan2018mapping}.
				
				


\section{Own contribution}
\begin{figure}
  \includegraphics[width=\linewidth]{latex_files/figures/seminar_own_contri.pdf}
  \caption{Flowchart for automated preference elicitation from social media profile}
  \label{fig:table3}
\end{figure}

Even though a lot of literature has been aimed at deriving the traveller personality and recommending travel destinations to the travellers from their pictures that are publicly available from their social media or from the list of destinations that are rated by experts, exploring an individual's publicly available activity on social media to suggest travel destinations to him remains a significantly unexplored problem.

One possible approach to solve this could be gathering all the publicly available activities for a particular user from his social media profile, say Facebook, and then training separate competitive models for each of his set of 'likes', 'comments', 'interests' and 'check-ins'. Clustering is then performed on each of the models to come up with a set of novel clusters for every model. 

Here we assume for simplicity that the data points for the centroids of clusters for the seven factor model \cite{sertkan2018mapping} is already available and known. The distances between centroid of each cluster from  the competitive models and the seven factor model are compared and the cluster from every model which is most similar to the seven factor model is considered. This results in the formation of four different clusters points which are most similar to one of the clusters from the seven factor model.

These results are now combined to generate a recommendation for destination to the traveller, that matches both his travel personality as well as his choices.
\section{Conclusion}

% \appendix
% %Appendix A
% \section{The Author}
% Ashmi Banerjee, currently a masters student in the department of Informatics at Technical University of Munich, received her Bachelor\textquotesingle s degree in Computer Science and Engineering from Heritage Institute of Technology, Kolkata, India.

% This paper was written as a part of an assignment during the Master-Seminar course 'Current Topics in Recommender Systems', organized by the Chair of
% Connected Mobility,during the winter semester 2018-19, at the Technical University of Munich.

% \section{Acknowledgement}